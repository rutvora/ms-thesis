\subsection{Network side-channel mitigation using network middlebox}
\label{subsec:future-work-netshaper}

\paragraph{Reducing the middlebox latencies.}
The current prototype of NetShaper relies on the Linux kernel to provide TCP/UDP. 
While this approach allows NetShaper to be deployed with superuser privileges, it adds to the end-to-end latency as the application data traverses the Linux kernel network stack multiple times.
Reducing this latency is paramount for real-time applications.
One approach to this would be to use kernel bypass mechanisms such as DPDK \cite{dpdk} and F-Stack \cite{fstack} to reduce the latency induced by the context switches between the kernel and user-space.

\paragraph{Multi-threaded \textit{Prepare} phase.}
The current implementation is limited by the design of the \textit{Prepare} thread, which requires that there is only one \textit{Prepare} thread.
However, this creates a bottleneck when scaling to higher network bandwidths, such as 100 Gbps, as only one \textit{Prepare} thread would be available to prepare and enqueue the data to the \textit{QUIC Worker} threads.
This issue can be alleviated by splitting the \textit{Prepare} thread further into multiple threads, each of which can enqueue the data to the \textit{QUIC Worker} for transmission.
One needs to take care of the locks between the various \textit{Prepare} threads and \textit{QUIC Worker} threads to ensure that none of the \textit{QUIC Worker} threads starts transmitting before all of the \textit{Prepare} threads have enqueued their data.

