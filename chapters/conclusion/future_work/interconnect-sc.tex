\subsection{Side Channels in PCIe}
\label{subsec:future-work-side-channels}

\paragraph{Inferring the size of the data transfer.}
As our work demonstrated, the size of the data transferred by the victim can affect either the latency of the delay or the duration of the delay observed by the attacker.
The attacker can use this information to determine the size of the data being transferred by the victim.
The size of the data transfer may help the adversary determine information about the workload the victim is running.
For example, the victim might be training or inferring from large machine-learning models that cannot fit in the GPU's memory.
Such models must be transferred or pre-fetched layer-by-layer from the CPU's memory.
An adversary can observe the size of these transfers and determine the size of each layer.
Furthermore, the duration for which the attacker does not observe any contention may correspond to the execution time of a previously fetched layer, which can depend on the layer type.
Based on such information, the adversary can reverse engineer the architecture of the machine-learning model the victim is using.

\paragraph{Side Channels in CXL.}
As CXL uses the same physical layer as PCIe, it can also be vulnerable to similar attacks.
Furthermore, CXL 2.0 and the versions after that support more than one host communicating with the same CXL device or endpoint \cite{cxl_2}.
This would further aggravate the problem, as the adversary does not need to be co-located on the same machine as the victim.

\paragraph{Defending against side-channels in PCIe.}
As a future research direction, one can focus on defending such side-channel attacks in PCIe.
As PCIe behaviour is largely dependent on the PCIe controller, software-based approaches such as NetShaper cannot be deployed.
Another challenge is providing mitigation that either does not negatively impact the latency or has a modest latency overhead for any application.