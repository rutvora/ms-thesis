% \section{Introduction}
% \label{sec:netshaper-intro}

Encryption, which has become the de-facto mechanism for protecting communication over the internet, does not conceal the shape of the traffic.
Prior work has demonstrated that in many applications, this traffic shape has a strong correlation with the data being transmitted and, as such, is a potential vector for network side-channel attacks.
For example, webpages on the internet access different resources such as CSS, JavaScript and images in a unique pattern.
This access pattern can be fingerprinted \cite{gong2010fingerprinting, bhat2019varcnn, wang2014supersequence}.
Similarly, streaming videos on the internet can also be fingerprinted.
Most streaming videos rely on the DASH standard \cite{dash2013} in which the videos are split into five-second segments, which are compressed individually and transmitted in a burst of traffic every five seconds, thus creating a unique fingerprint \cite{schuster2017beautyburst}.


While many solutions have been proposed to mitigate network side-channel attacks, few provide any deployable mitigation system.
Of the ones that do provide deployable systems, many require non-trivial modifications to the end host and either do not support or do not scale well with multiple users \cite{cai2014csbuflo, cherubin2017llama, luo2011httpos, smith2022qcsd, wang2017walkie, mehta2022pacer}.
Others are either specific to a given application or do not adapt to the traffic patterns of different applications \cite{barradas2017deltashaper, cherubin2017llama, luo2011httpos}.
In addition, these systems do not have a modular design that can be easily extended to support additional types of traffic and additional protocols.

This chapter presents the NetShaper system, a modular, portable, scalable network side-channel mitigation system.
NetShaper is a transport layer (L4) proxy tunnel that can be integrated with any network stack and within any node.
Our system requires only that the end-host change their operating system's or browser's proxy configuration.
NetShaper's modular architecture allows easy modification of any system sub-component to support additional protocols or alternative implementations.
To support and scale with multiple clients, NetShaper relies on the QUIC protocol \cite{quic_rfc}, assigning a QUIC stream to every client-server pair.

The rest of the chapter is organised as follows:
In \Cref{sec:netshaper-background}, we first provide a background on network side-channel attacks, how differential privacy can be used to mitigate such attacks, and how the QUIC protocol, which is fundamental to the scalability of NetShaper, works.
In \Cref{sec:netshaper-threat-model}, we outline the Threat Model for which NetShaper provides mitigation.
We then present the design of NetShaper's QUIC-based traffic shaping tunnel in \Cref{sec:netshaper-designing-traffic-shaping-tunnel}.
\Cref{sec:netshaper-middlebox-implementation} presents the system implementation of NetShaper.
We evaluate the performance of NetShaper and provide the results in \Cref{sec:netshaper-evaluation}.
We then discuss the limitations of NetShaper in \Cref{sec:netshaper-limitations}.
Finally, we conclude the chapter by providing a comprehensive view of the related work and how NetShaper compares to them in \Cref{sec:netshaper-related-work}.

