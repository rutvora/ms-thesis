\section{System Design Requirements}
\label{sec:netshaper-system-requirements}

Before delving into the system design of NetShaper, we first outline the requirements that the system should fulfil.
First, in any given window $W$, the system should be able to obtain the size of the payload in the buffering queue, with noise added to it. 
That is, the system should be able to complete the execution of $f_{DP}(S, t_{start}, t_{start} + W)$.
In the same window, the system should also be able to send out the payload, with padding, if necessary, such that the total data sent out, $b_{out}$, is equal to the noised size. 
However, it is sufficient to be able to queue $b_{out}$ bytes to be sent out, even if the actual transmission goes beyond the window $W$, as long as any delays were not caused by the payload coming in or already present in the buffering queue. 
The reason for this is the post-processing property of DP that we outlined in \Cref{subsubsec:netshaper-background-framework-dp}.

Second, the payload and the padding should be indistinguishable to any observer observing the outbound packet stream.
Hence, the payload and padding should both be subject to the same congestion control, re-transmission, loss recovery, and other network behaviour.
In addition, the outbound transmission should provide the same or a higher level of reliability than the applications using this system expect.

Finally, the system should be modular so that modifications to any one sub-component do not require changes in the other components.
The system should also be portable and easily deployable, requiring none to minimal changes on the end hosts where the applications are running.
These goals ascertain that the system is easy to adopt and deploy and can easily be modified per the deployer's requirements.

\endinput

1. Complete DP measurement within the window W
2. Data and Dummy should be indistinguishable
3. Should provide the same level of reliability the application expects
4. Should be modular for easy modification to sub-components
5. Should be portable and easily deployable, with minimal modifications of the end-hosts.
