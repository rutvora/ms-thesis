\subsection{The QUIC protocol}
\label{subsec:netshaper-background-quic}

QUIC is a connection-oriented transport layer protocol that can be deployed on top of UDP.
It is now standardised under RFC 9000 \cite{quic_rfc}.
It provides many features that are similar to TCP, such as flow control, loss recovery, and congestion control. 
However, it also alleviates some problems that TCP encounters.
For example, QUIC enforces encryption in the initial handshake, thus ensuring that traffic is always encrypted.
In addition, a QUIC connection can consist of multiple dynamically created streams, each of which can act as an independent byte stream. 
This alleviates the problem of head-of-line blocking faced by TCP, ensuring that one blocked stream does not affect others, especially when proxying the data of multiple clients to a single server via a single connection.
Each stream has a unique header containing the stream ID and the stream type.
Multiple such streams, with their headers, can be a part of a single encrypted QUIC packet.
This ensures that an observer can not determine the number of streams being transmitted in a QUIC packet just by observing the encrypted packet or traffic stream.


