\subsection{NetShaper Framework}
\label{subsec:netshaper-background-framework}

NetShaper \cite{sabzi2024netshaper} is both the name of a differential-privacy-based framework for mitigating network side-channel attacks and the system that utilises that framework. 
Here, we provide a brief description of the NetShaper Framework.

\subsubsection{Differential Privacy}
\label{subsubsec:netshaper-background-framework-dp}
NetShaper's framework relies on Differential Privacy (DP) to provide robust, mathematical guarantees for defending against side-channel attacks in internet networks.
DP is a framework originally proposed by \citet{dwork2006differential} for releasing usable aggregate metrics regarding a dataset while limiting the leakage of information about individual data points in that dataset.
In other words, DP bounds the probability of an observer being able to infer if an individual data point was used to generate the aggregate metric the observer has access to.
We provide the formal definition of DP in \Cref{def:dp}. \\
DP has two main parameters: \\
\textbf{$\epsilon$: } It is the value that controls the trade-off between privacy and usefulness of the aggregate metric.
A lower value implies that an individual data point influences the aggregate result less, but also decreases the accuracy of the aggregate metric, as the metric relies less on the individual data points.
Conversely, a higher value implies that an individual data point influences the aggregate results more, and hence has a higher probability of an observer determining their presence or absence based on the aggregate metric.\\
\textbf{$\delta$: } It is the probability with which a given differentially-private mechanism may fail.

\begin{definition}[Differential privacy]
  \label{def:dp}
  A randomized algorithm $A_{DP}$ is $(\varepsilon, \delta)-DP$ if for all ${S} \subseteq Range(A_{DP})$ and for all datasets $D, D'$ that differ on a single element, we have:
  \begin{equation*}
    \Pr[A_{DP}(D) \in S] \leq \exp(\varepsilon)\Pr[A_{DP}(D') \in S] + \delta
  \end{equation*}
\end{definition}

For any aggregate algorithm $A$, we can make an $(\varepsilon, \delta)-DP$ algorithm by adding some noise to the output of $A$.
There are many different mechanisms to add the noise $\eta$, but most commonly, $\eta$ is a function of $\varepsilon, \delta$ and other dataset-dependent parameters. 
Mathematically, $A_{DP}(D) = A(D) + \eta(\varepsilon, \delta, ...)$

DP has a few useful properties: 
\textbf{1) Post Processing:} Any further operations or processing on the output of an $(\varepsilon, \delta)-DP$ algorithm is also $(\varepsilon, \delta)-DP$.
This condition holds true as long as the operations carried out do not involve any auxiliary knowledge of the dataset.
\textbf{2) Composition:} It is possible to quantify the total privacy loss when multiple queries are issued to the same $(\varepsilon, \delta)-DP$ algorithm.

\subsubsection{Applying DP on network streams}
\label{subsubsec:netshaper-background-framework-applying-dp}
NetShaper uses DP to mitigate network side-channel attacks. 
To achieve this, NetShaper relies on two key ideas: 
First, NetShaper represents network traffic streams as datasets to which DP can be applied.
As DP applies to datasets, we must first represent a network stream as a dataset.
A network stream $S$ can be represented as a sequence of packets $P_i^S$, where each packet is associated with its length and the timestamp at which it was encountered (i.e. $P_i^S = (l_i^S, t_i^S$).
Hence, $S$ is the dataset consisting of the traffic shape, which can be used by an attacker to carry out a side-channel attack, as outlined in \Cref{subsec:interconnect-sc-background-side-channels}.
As the network stream can potentially be long, NetShaper models the DP guarantees in windows of fixed length $W$ and can use composition to calculate the privacy loss across multiple windows.
Second, NetShaper relies on a buffering queue to control the shape of the traffic visible to the attacker within each window $W$ by discretising time into $W$-sized windows.
At the beginning of each discrete window, NetShaper checks the size of the buffer and adds DP noise to the size to determine the amount of data to be sent out in that window.


So, initially, the attacker was able to query a function $f(S, t_{start}, t_{end})$ to obtain the size and time of the packet that was transmitted by the victim. 
However, with the application of DP on the network stream, the attacker can now only query a new function $f_{DP}(S, t_{start}, t_{end})$.
$f_{DP}$ is a differentially private function ensuring that the probability of leaking individual entries of the dataset is bounded, and where $t_{end} - t_{start} = kW, k \in N$.

\endinput

Note: Should we add stuff about sensitivity? (I don't think it's necessary for my thesis)