\section{Generating contention via DMA}
\label{sec:interconnect-sc-dma}

In this section, we discuss how an adversary can use direct memory access (DMA) to create contention in the buffers of a PCIe controller.
We revisit \textbf{RQ1} to determine if an adversary can use this approach to observe the victim's traffic pattern. 
In addition, we also provide an answer to \textbf{RQ2} by demonstrating that an adversary can create congestion on a non-hierarchical PCIe 4.0 link.

A DMA controller can generate memory addresses and initiate memory \textit{load} or \textit{store} instructions that are executed in parallel, independent of the execution units of the CPU.
The DMA controller receives commands specifying the source and destination addresses and the size of the data to copy.
Once the data transfer is complete, it issues a notification of completion to the command issuer.
As such, DMA controllers like those on Nvidia GPUs provide a mechanism to measure only the completion time of the entire DMA transfer, not the completion time of individual \textit{load} or \textit{store} instructions.

\subsection{Challenges}
\label{subsec:interconnect-sc-dma-challenges}

\subsubsection{Trade-off between small vs large DMA transfers}
\label{subsubsec:interconnect-sc-dma-challenges-trade-off-small-v-large-tx}

Since DMA controllers only provide the completion time of an entire DMA transfer, the adversary faces a trade-off when selecting the transfer size. 
Small transfers enable fine-grained measurements of the presence/absence of victim traffic but cannot saturate a non-hierarchical PCIe link. 
In contrast, large transfers provide coarse-grained measurements but may be able to saturate a non-hierarchical PCIe link, increasing the likelihood of interfering with the victim traffic.
To determine the appropriate transfer size, we profile the execution time and bandwidth utilisation of PCIe across different DMA transfer sizes.

To perform the DMA transfers, we rely on the \textit{cudaMemcpy} function provided by Nvidia CUDA.
We ensure that the memory on the CPU is pinned to the DRAM and can not be swapped, as the DMA controller requires this.
We measure the DMA transfer time using \textit{cudaEventRecord}, which uses a timer from the GPU to measure the time taken by GPU operations such as \textit{cudaMemcpy}
\footnote{While we can follow the approach taken by LockedDown \cite{side2022lockeddown} and use the CPU-side timers such as RDTSC(P), those measurements would include the time taken by the GPU driver to send the transfer information to the DMA controller.}.

% To perform the DMA transfers, we rely on the \textit{cudaMemcpy} function provided by Nvidia CUDA.
% \textit{cudaMemcpy} accepts the source and destination addresses (at least one of which should point to some GPU memory) and the size of the data to be transferred.
% It expects the address associated with the CPU memory (if any) to be pinned in the CPU DRAM (i.e. the memory is not swapped to disk at any point).
% If that is not the case, \textit{cudaMemcpy} first allocates a pinned memory, copies the data over to this memory, and then sends the corresponding commands to the DMA engine for initiating the transfer.

% In our experiments, we rely on memory that is already pinned in the CPU DRAM, thus avoiding the additional overhead of \textit{cudaMemcpy} copying the data to a newly allocated pinned memory.
% In addition, our experiments utilise \textit{cudaEventRecord} provided by CUDA to measure the execution time of \textit{cudaMemcpy}.
% However, \textit{cudaEventRecord} uses the timer from the GPU, which only has a granularity in the order of microseconds.
% In addition, \textit{cudaEventRecord} returns the timer as a floating point number in the order of milliseconds, adding to the inaccuracy of the timer.

As shown in \Cref{fig:bw-util-and-time-per-size}, the transfer time increases with the transfer size, while the transfer rate does not increase beyond 25GBps
\footnote{While the theoretical limit of PCIe 4.0 x16 is 32GBps, the achievable transfer rate is less due to the protocol overheads \cite{neugebauer2018understanding}.},
at a transfer size of 4MB.
Therefore, if an adversary aims to saturate the PCIe bandwidth, they can optimise the transfer size at 4MB to strike a balance between fine-grained measurements and saturating the PCIe bandwidth.

\begin{figure}[!htb]
    \centering
    \includegraphics[width=\columnwidth]{figures/interconnect-sc/dma/bw_util_and_time_per_size.png}
    \caption{PCIe bandwidth utilisation and time taken for the DMA transfer}
    \label{fig:bw-util-and-time-per-size}
    % 2025-03-06_16-40
\end{figure}

\subsection{Using DMA operations to observe presence/absence of victim traffic on PCIe}
\label{subsec:interconnect-sc-dma-evaluation}

Now that we know that the PCIe bandwidth can be saturated with 4MB transfers, we can use this information to craft an adversary that can create congestion in the non-hierarchical PCIe link by saturating the link.
We use the pseudo-code outlined in \Cref{lst:timing-victim-with-dma} to achieve this.
We assume the victim also transfers data to/from the GPU via DMA.

To evaluate our ability to detect victim traffic, we run the adversary for $10^4$ iterations.
At the same time, we execute the victim, which repeatedly (for $10^3$ iterations) performs a DMA transfer of either 4KB or 4MB
As before, we synchronise the execution of the critical code sections responsible for data transfers between the victim and the adversary. 

As shown in \Cref{fig:dma-contention-4mb}, the time it takes to complete the call to \textit{cudaMemcpy} increases in the presence of victim traffic and reduces once the victim stops transmitting.
As such, an adversary can determine the presence of victim traffic.

Furthermore, we also evaluate if the attacker can determine the presence/absence of victim traffic without saturating the PCIe bandwidth.
For this, we follow the same pseudo-code in \Cref{lst:timing-victim-with-dma} but set the transfer size of the adversary to 4KB.
As shown in \Cref{fig:dma-contention-4kb}, the adversary can still observe the presence or absence of the victim traffic.


\begin{minipage}{\textwidth}
    \lstinputlisting[language=Python]{code/interconnect-sc/timing-victim-with-dma.py}
    \captionsetup{type=lstlisting}
    \caption{Attacker code to detect presence of victim traffic via DMA operations}
    \label{lst:timing-victim-with-dma}
\end{minipage}

\begin{figure}
     \centering
     
     \begin{subfigure}[b]{\textwidth}
         \centering
        \includegraphics[width=\textwidth]{figures/interconnect-sc/dma/dma_contention_4MB.png}
        \caption{Attacker transfers 4MB}
        \label{fig:dma-contention-4mb}
     \end{subfigure}

    \hfill
     
     \begin{subfigure}[b]{\textwidth}
        \centering
        \includegraphics[width=\textwidth]{figures/interconnect-sc/dma/dma_contention_4KB.png}
        \caption{Attacker transfers 4KB}
        \label{fig:dma-contention-4kb}
     \end{subfigure}

     \caption{The attacker can observe the presence of victim traffic both with and without saturating the PCIe bandwidth}
    \label{fig:dma-contention}
    % 2025-03-06_17-18
\end{figure}


However, unlike the contention observed in \Cref{fig:cpu-store-victim-observation}, the duration of contention observed by the attacker here remains constant regardless of the transfer size.
Hence, it remains unclear whether the contention occurs in the PCIe controller or if the observation results from the DMA controller following a round-robin-like schedule for multiple scheduled DMA transfers.
To verify this, we rerun the same algorithm where the attacker transfers 4KB, but with the victim transferring 64MB or 256MB of data.
As shown in \Cref{fig:dma-contention-round-robin}, the duration for which the attacker faces a delay remains constant.
Still, the latency the attacker observes increases when the victim transfers more data.
These observations suggest that the contention arises from the DMA controller's round-robin-like scheduling policy, where the attacker's transfer is delayed for the duration of the victim's transfer.

\begin{figure}[!htb]
    \centering
    \includegraphics[width=\textwidth]{figures/interconnect-sc/dma/dma_contention_round_robin.png}
    \caption{DMA contention is due to the DMA controller's round-robin-like scheduling policy}
    \label{fig:dma-contention-round-robin}
    % 2025-03-10_09-51
\end{figure}