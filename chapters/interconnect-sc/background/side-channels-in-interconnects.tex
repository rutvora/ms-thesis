\subsection{Side Channels in Interconnects}
\label{subsec:interconnect-sc-background-side-channels}

Conceptually, carrying out a side-channel interconnect is very similar to carrying out a buffer congestion-based attack on internet networks, like the one presented by \citet{schuster2017beautyburst}. 
The attacker creates congestion on a buffer in the communication path between the host (CPU) and the peripheral (e.g. GPU), such as the Rx/Tx buffers in the transaction layer implementation of the PCIe controller.
The attacker then observes the delay in transmitting their own traffic.
This delay would be proportional to the amount of data transferred by the victim, which would help the attacker gain insight into the victim's traffic shape.

However, a few differences between interconnects and traditional networks make it challenging to implement such an attack. \\
\textit{First}, The bandwidth of interconnects like PCIe is at least an order of magnitude higher than that of internet networks.
While a typical bandwidth in an internet network may be 1-10Gbps ( = 0.125 - 1.25 GBps), PCIe 4.0 can operate at a rate of 32GBps.
The high bandwidth makes it non-trivial to saturate the PCIe link. \\
\textit{Second}, latency in PCIe is at least three orders of magnitude smaller than that of internet networks.
Typically, internet latency is measured in milliseconds, while PCIe latency is measured in hundreds of nanoseconds to microseconds.
The low latency makes it difficult for the attacker to measure the delays accurately. \\
\textit{Third}, unlike internet networks, PCIe does not extend outside the machine of the host CPU.
This further adds to the challenge that the attacker needs to co-locate with the victim application on the same machine. \\
\textit{Fourth}, the PCIe protocol has a fixed behaviour depending on the type of transaction.
The attacker can not craft their own packets and can not change how the hardware issues the PCIe transactions.
% Most transaction types can not generate enough traffic to even come close to saturating the bandwidth of PCIe. 
% While saturating the bandwidth may not be necessary to create a side-channel attack, it would be useful if there is always at least one attacker packet in the buffer whenever the victim might transmit.
% This would ensure that the attacker can observe a delay in their own packets for each packet or set of packets the victim transmits.

\begin{comment}
https://www.linkedin.com/pulse/pci-express-primer-3-transaction-layer-simon-southwell/
\end{comment}