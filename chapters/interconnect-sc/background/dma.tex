\subsection{Direct Memory Access}
\label{subsec:interconnect-sc-background-dma}

Direct memory access (DMA) enables data transfer between the CPU memory and the peripheral devices without the need to involve the execution units of the CPU.
A DMA controller is a purpose-built hardware that massively parallelises data movement operations between the CPU memory and the peripheral devices.
The DMA controller obtains instructions regarding the source and destination address, and size of the data to be transferred.
The DMA controller then takes control of the system buses required to perform this data transfer based on the configured mode of operation.

\subsubsection{Modes of operation}
DMA has three modes of operation.
\textbf{Burst Mode:} In this mode, the DMA controller transfers a block of data in one contiguous sequence before handing over the control of the system bus back to the CPU.
During the transmission, the CPU can not perform any action that requires control over the system bus.
Depending on the total size of the transfer and the size of the block, this action may happen multiple times before all of the data is transferred.
For example, when transferring multiple 4KB pages, the DMA controller may transfer each page as an individual transfer block.
This mode is used for copying a large amount of data, as it provides the fastest transfer rate for DMA transfers.
Peripherals such as GPUs, FPGAs or custom accelerators can use this mode to transfer large amounts of data to/from the CPU DRAM.

\textbf{Cycle Stealing Mode:} In this mode, the CPU maintains control of the system bus until the peripheral device is ready to transmit the data.
Once the device is ready, the CPU transfers control of the bus to the peripheral device for one cycle.
This mode is used when the peripheral may take a lot of time to prepare the data and the amount of data is small.
While this approach ensures that the CPU is not blocked for a long duration, it reduces the DMA transfer rate.
Systems that require real-time processes, such as audio processing systems, use this mode to ensure that time-critical CPU operations are blocked predictably.

\textbf{Interleaving Mode:} In this mode, the CPU maintains control of the system bus and only releases it when it does not need it.
As such, the device has to wait to perform the data transfer even when the data is prepared.
While this approach ensures that the CPU is never blocked, it results in a very slow DMA transfer rate.
This approach was used by older peripheral devices, such as a floppy disk controller, to lazily transfer data from the disk to the CPU DRAM.