\section{Generating contention via CPU instructions}
\label{sec:interconnect-sc-store-ops}

There are two major ways one can generate traffic on the PCIe link: 
1) CPU load/store instructions on memory-mapped PCIe device memory, I/O or configuration address space, and 
2) Direct memory access (DMA) to copy data between the PCIe device memory and the DRAM.
In this section, we discuss how to use CPU instructions to generate the traffic and defer the discussion of generating traffic via DMA to \Cref{sec:interconnect-sc-dma}.

An adversary can generate traffic from the CPU by mapping a PCIe device resource to an application running on the CPU and issuing load or store operations on the mapped memory.
Each load/store operation will be buffered in the PCIe controller until a transaction layer packet (TLP) is generated and ready to be transmitted.
The actual transmission time of the packet is influenced by two factors:
1) The ordering of the packets, as outlined in \Cref{tab:pcie-transaction-ordering-rules}, and
2) The number of pending packets that are not subject to any ordering rules relative to the current packet.
If the packets do not have any relative ordering rule, it is reasonable to assume that the controller follows a simple first come, first serve policy.
As such, any packet generated by the adversary can either be delayed due to the presence of existing traffic in the buffer or not be delayed, as there is no traffic in the buffer.
This allows the attacker to observe the presence or absence of victim traffic without necessarily saturating the PCIe bandwidth.


To observe the victim traffic, the attacker always needs to have at least one packet pending in the PCIe controller, which can be delayed by the presence of the victim traffic.
The attacker also needs to ensure that the packet can not be delayed because of their own traffic.
To achieve these goals, the attacker needs to determine the correct combination of the PCIe resource and instruction, as different instructions on different resources can lead to different PCIe transactions and transaction ordering rules.


\subsection{Challenges}
\label{subsec:interconnect-sc-store-ops-challenges}

Multiple store operations can be asynchronously issued on most modern processors as they support out-of-order execution. 
However, the same out-of-order execution feature makes it difficult for the attacker to accurately measure a store operation's time since the timer instruction can also be executed out of order.
To measure time accurately, most applications rely on issuing a fence instruction
\footnote{A fence instruction imposes ordering constraints, thus ensuring that any instruction after the fence is not executed before \textit{all} instructions before the fence are executed}.
However, this solution would not work for the attacker as the fence would not allow the next store instruction to be issued in parallel to the previous one, negating the benefit of using a posted (async) transaction.
Hence, to measure the time of repeated instructions (store ops) that are executed out-of-order, we need to understand how the CPU core manages out-of-order execution.

\subsubsection{Reverse Engineering the CPU architecture}
\label{subsubsec:interconnect-sc-store-ops-challenges-reverse-engineering}

\subsubsection{Measuring time of asynchronous CPU instructions}
\label{subsubsec:interconnect-sc-store-ops-challenges-measuring-time}

\subsubsection{Microcode Updates}
\label{subsubsec:interconnect-sc-store-ops-challenges-microcode-updates}
\subsection{Using \textit{store} operations to observe presence/absence of victim traffic on PCIe}
\label{subsec:interconnect-sc-store-ops-measuring-time}
Now, we have a mechanism to issue multiple \textit{store} instructions in parallel while timing the completion of an individual (or a small group of) \textit{store} instructions.
We can use this mechanism and the threshold of $N = 81$ to observe the presence or absence of victim traffic in the PCIe controller.
To achieve this, we execute the pseudo-code outlined in \Cref{lst:timing-victim-with-stores} while the victim transfers a large amount of data to or from the GPU via DMA
\footnote{Since most GPU-based applications rely on the DMA controller for efficient transfers, we assume that the victim uses DMA.}.

To evaluate our ability to detect victim traffic, we run the adversary for $10^5$ iterations.
At the same time, we execute the victim, which repeatedly (for $10^4$ iterations) performs a DMA transfer of either 4KB or 4MB.
The 4KB transfer does not saturate the PCIe link, while the 4MB transfer saturates the PCIe link (see \Cref{fig:bw-util-and-time-per-size}).
Additionally, we synchronise the execution of the critical code sections responsible for data transfers between the victim and the adversary. 
This synchronisation allows us to measure whether we can detect the presence or absence of victim traffic, even though such perfect alignment is unrealistic for a side-channel attack.

As shown in \Cref{fig:cpu-store-victim-observation}, the execution time of the individual \textit{store} operations by the adversary increases in the presence of victim traffic, irrespective of whether the victim is saturating the PCIe link or not.
Without a victim, the \textit{store} instruction takes $<15~cycles$ to complete. However, in the presence of victim traffic, the execution time increases to $>15~cycles$ and drops again once the victim is done transmitting.
As such, an adversary can determine the presence/absence of victim traffic.
The adversary observes a fixed delay in the execution time of the \textit{store} instruction irrespective of the size of the data being transferred by the victim.
However, the duration for which the adversary observes the delay increases with the amount of data the victim transfers.
This is the case because the victim transfers the data using \textit{burst mode} of transfer, where the burst size is fixed.


\begin{minipage}{\textwidth}
    \lstinputlisting[language=Python]{code/interconnect-sc/timing-victim-with-stores.py}
    \captionsetup{type=lstlisting}
    \caption{Attacker code to detect presence of victim traffic via \textit{store} instructions}
    \label{lst:timing-victim-with-stores}
\end{minipage}

\begin{figure}[!htb]
    \centering
    \includegraphics[width=\columnwidth]{figures/interconnect-sc/store-ops/cpu_store_victim_observation.png}
    \caption{Observing the presence of victim DMA traffic using \textit{store} instructions. The victim does $10^4$ iterations of transferring 4KB/4MB.}
    \label{fig:cpu-store-victim-observation}
    % 2025-03-06_14-20
\end{figure}

\subsection{Evaluation}
\label{subsec:interconnect-sc-store-ops-evaluation}
