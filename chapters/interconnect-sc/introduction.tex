% \section{Introduction}
% \label{sec:interconnect-sc-introduction}

Interconnects facilitate high-speed and low-latency communication within and between processing units, memory, and peripheral devices like the GPU.
While the evolution of interconnects has prioritised speed and latency, security has received less attention.
On-chip interconnects do not have any encryption even today, while off-chip interconnects have only recently started focusing on encrypting the data in transit.
For example, data transmitted off-chip to memory or another NUMA node is now encrypted when the user is using Intel's SGX/TDX or has AMD's SME or SEV enabled \cite{intel_upi_encryption, amd_gen_5_arch}.
PCIe, which is used to communicate with peripheral devices, only introduced encryption with the PCIe v6.0 specifications 
\footnote{PCIe 5.0 devices are only recently coming to market, and PCI-SIG has released an Engineering Change Notification (ECN) to optionally enable integrity and data encryption (IDE) for PCIe 5.0 \cite{pcie_ide_v5_ecn}}
\cite{pcie_ide_v6}.
While the recent focus on interconnect security is on encryption, little focus has been given to network-side channel attacks on interconnects.

Recent research has demonstrated that creating congestion in PCIe can help an adversary determine which website a user is visiting \cite{tan2021invisible, side2022lockeddown}, what the user is typing in a GUI environment \cite{tan2021invisible}, what machine learning model they are training \cite{tan2021invisible}, or what resources they are using on an FPGA connected via PCIe \cite{giechaskiel2022cross}.
Recent work has also shown that congestion in PCIe can be used as a covert channel to exfiltrate data from a secure environment to a collaborator outside the environment \cite{giechaskiel2022cross, khaliq2021timing}.
However, prior work relies on PCIe 3.0 and often has a hierarchical topology where there is a shared PCIe switch or PCH.
In addition, they rely on the adversary's ability to saturate the PCIe bandwidth, hence creating congestion.

This chapter answers two important research questions:
\textbf{RQ1:} Is it possible to create contention in PCIe links without the adversary saturating the PCIe bandwidth?
\textbf{RQ2:} Is it possible to create congestion on a non-hierarchical PCIe link in faster PCIe v4.0?

The rest of the chapter is organised as follows.
We begin by providing some relevant background on PCIe, the CPU architecture we work with, and interconnect side-channels in general in \Cref{sec:interconnect-sc-background}. 
Then, we outline our threat model and our setup in \Cref{sec:interconnect-sc-threat-model} and \Cref{sec:interconnect-sc-setup}, respectively.
We answer the research questions outlined above in \Cref{sec:interconnect-sc-store-ops} and \Cref{sec:interconnect-sc-dma}.
Finally, we discuss the limitations of our approach in \Cref{sec:interconnect-sc-limitations} and how our approach compares to related work in \Cref{sec:interconnect-sc-related-work}.
