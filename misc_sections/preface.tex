%% The following is a directive for TeXShop to indicate the main file
%%!TEX root = thesis.tex

\chapter{Preface}


This thesis contains text and material from the following publication~\cite{sabzi2024netshaper}:
\begin{itemize}
  \item ``NetShaper: A Differentially Private Network Side-Channel Mitigation System''.
  \\
  Amir Sabzi, Rut Vora, Swati Goswami, Margo Seltzer, Mathias Lécuyer, Aastha Mehta. \textit{USENIX Security '24}
\end{itemize}

Following is the origin of all text, figures, and tables presented in this thesis.

\begin{itemize}
  \item \textbf{\Cref{ch:Introduction}}: The introduction is written by me. 
  
  \item \textbf{\Cref{ch:netshaper}}: This chapter is partially derived from the NetShaper paper \cite{sabzi2024netshaper}, and extends the evaluations presented in the paper.
  \begin{itemize}
      \item \Cref{subsec:netshaper-background-network-side-channels} and \Cref{subsec:netshaper-background-framework} are derived from the paper and were written by my collaborators Amir Sabzi, Aastha Mehta and Mathias L\'{e}cuyer. The rest of the background sections are written by me.
      \item \Cref{sec:netshaper-threat-model}, \Cref{sec:netshaper-designing-traffic-shaping-tunnel}, and \Cref{sec:netshaper-middlebox-implementation} are largely written by me, and derived from the NetShaper paper.
      \item \Cref{sec:netshaper-evaluation} is written by me. A collaborator, Arun Balamurali, helped set up the test bed and run the experiment in \Cref{subsec:netshaper-evaluation-bw}.
      \item \Cref{sec:netshaper-limitations} and \Cref{sec:netshaper-related-work} are written by me and share content with the paper.
  \end{itemize}
  
  \item \textbf{\Cref{ch:interconnect-sc}}: This chapter is entirely written by me.
  
  \item \textbf{\Cref{chap:conclusion}}: This chapter is entirely written by me.

\end{itemize}


Additionally, I acknowledge the use of text and grammar editing tools (Grammarly \cite{grammarly}) and generative artificial intelligence (GitHub Copilot \cite{github_copilot}, ChatGPT \cite{chatgpt}, and Perplexity \cite{perplexity}). 
These tools were used only for clearer text and to help with scripts for generating figures.