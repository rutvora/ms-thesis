\subsection{PCIe transactions}
\label{sec:pcie-transactions-background}

PCIe supports three types of transactions:\\
\textbf{Posted:} Transactions where no response is issued or expected. 
These transactions are also asynchronous, and hence allow multiple transactions of the same type to be in flight at the same time.\\
\textbf{Non-posted:} Transactions where a response is required. 
These types of transactions are also synchronous.
As such, one can not execute multiple of these simultaneously.\\
\textbf{Completions:} The completion of a previous non-posted transaction.\\


\begin{table}[h]
    \centering
    \begin{tabular}{|l|l|p{0.65\textwidth}|}
        \hline
        \textbf{Transaction} & \textbf{Type} & \textbf{Description} \\ 
        \hline
        Memory Read         & Non-Posted & Read from a memory-mapped address space \\ 
        Memory Write        & Posted     & Write to a memory-mapped address space  \\ 
        I/O Read            & Non-Posted & (Legacy PCI) Read from the I/O address space \\ 
        I/O Write           & Non-Posted & (Legacy PCI) Write to the I/O address space \\ 
        Configuration Read  & Non-Posted & Read control and status registers of the PCIe interface \\ 
        Configuration Write & Non-Posted & Write control and status registers of the PCIe interface \\  
        Message             & Posted     & Convey information that isn't an access to an addressable space (e.g. Interrupts, Power Management, Error Signalling, Vendor-defined messaging). \\
        Completion          & Completion & Response to all non-posted transactions \\ 
        \hline
    \end{tabular}
    \caption{PCIe transaction types}
    \label{tab:pcie-transaction-types}
\end{table}

\subsubsection{Challenge: Measuring the time of PCIe transactions}

\Cref{tab:pcie-transaction-types} outlines the different PCIe transactions and their types.
We can see that most PCIe transactions are Non-posted.
Only memory writes and messages are posted and hence asynchronous.
Asynchronous transactions such as a memory write (i.e. a \textit{store} instruction executed on memory-mapped PCIe endpoint memory region) would be more useful for carrying out a side-channel attack on PCIe.
This is because asynchronous transactions enable the attacker to have one transaction always pending to be sent out, regardless of the completion of the previous transaction(s).

However, measuring the completion time of an individual asynchronous transaction becomes more challenging as they are executed out-of-order. 
The naive solution of having a memory fence after each \textit{store} instruction would not work, as the memory fence would not allow the next \textit{store} instruction to be issued in parallel to the previous one, negating the benefit of using an asynchronous transaction.

\endinput

https://www.linkedin.com/pulse/pci-express-primer-3-transaction-layer-simon-southwell/
