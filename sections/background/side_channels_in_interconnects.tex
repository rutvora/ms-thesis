\section{Side Channels in Interconnects}
\label{sec:interconnect-sc-bg}
Conceptually, carrying out a side-channel interconnect is very similar to carrying out such an attack on internet networks. 
The attacker creates congestion on a buffer in the communication path between the host (CPU) and the peripheral (e.g. GPU) and observes their own delay.
This delay would be proportional to the amount of data transferred by the victim, which would help the attacker gain insight into the victim's traffic shape.

However, a few differences between interconnects and traditional networks make it challenging to implement such an attack. \\
\textit{First}, The bandwidth of interconnects like PCIe is at least an order of magnitude higher than that of internet networks.
While a typical bandwidth in an internet network may be 1-10Gbps ( = 0.125 - 1.25 GBps), PCIe 4.0 can operate at a rate of 32GBps.
The high bandwidth makes it non-trivial to saturate the PCIe link. \\
\textit{Second}, latency in PCIe is at least three orders of magnitude smaller than that of internet networks.
Typically, internet latency is measured in milliseconds, while PCIe latency is measured in microseconds.
The low latency makes it difficult for the attacker to measure the delays accurately. \\
\textit{Third}, unlike internet networks, PCIe does not extend outside the machine of the host CPU.
This further adds to the challenge that the attacker needs to co-locate with the victim application on the same machine. \\
\textit{Fourth}, the PCIe protocol has a fixed behaviour depending on the type of transaction. 
Most transaction types can not generate enough traffic to even come close to saturating the bandwidth of PCIe. 
While saturating the bandwidth may not be necessary to create a side-channel attack, it would be useful if there is always at least one attacker packet in the buffer whenever the victim might transmit.
This would ensure that the attacker can observe a delay in their own packets for each packet or set of packets the victim transmits.

We provide a list of the transactions PCIe supports in \Cref{tab:pcie-transaction-types}, and a description of what each PCIe transaction type entails below: \\
\textbf{Posted:} Transactions where no response is issued or expected. 
These transactions are also asynchronous and hence allow multiple transactions of the same type to be in flight at the same time.\\
\textbf{Non-posted:} Transactions where a response is required. 
These types of transactions are also synchronous.
As such, one can not execute multiple of these simultaneously.\\
\textbf{Completions:} The completion of a previous non-posted transaction.\\

\begin{table}[htb]
    \centering
    \begin{tabular}{|l|l|p{0.65\textwidth}|}
        \hline
        \textbf{Transaction} & \textbf{Type} & \textbf{Description} \\ 
        \hline
        Memory Read  & Non-Posted & Read from a memory-mapped address space \\ 
        Memory Write & Posted     & Write to a memory-mapped address space  \\ 
        I/O Read     & Non-Posted & (Legacy PCI) Read from the I/O address space \\ 
        I/O Write    & Non-Posted & (Legacy PCI) Write to the I/O address space \\ 
        Config Read  & Non-Posted & Read control and status registers of the PCIe interface \\ 
        Config Write & Non-Posted & Write control and status registers of the PCIe interface \\  
        Message      & Posted     & Conveys additional information (e.g. Interrupts, Power Management, Error Signalling, Vendor-defined messaging). \\
        Completion   & Completion & Response to all non-posted transactions \\ 
        \hline
    \end{tabular}
    \caption{PCIe transaction types}
    \label{tab:pcie-transaction-types}
\end{table}


\subsection{Challenge: Measuring the time of PCIe transactions}

In \Cref{tab:pcie-transaction-types}, we can see that most PCIe transactions are Non-posted.
Only memory writes and messages are posted and hence asynchronous.
Asynchronous transactions such as a memory write (i.e. a \textit{store} instruction executed on memory-mapped PCIe endpoint memory region) would be more useful for carrying out a side-channel attack on PCIe.
This is because asynchronous transactions enable the attacker to have one transaction always pending to be sent out, regardless of the completion of the previous transaction(s).

However, measuring the completion time of an individual asynchronous transaction becomes more challenging as they are executed out-of-order. 
The naive solution of having a memory fence after each \textit{store} instruction would not work, as the memory fence would not allow the next \textit{store} instruction to be issued in parallel to the previous one, negating the benefit of using an asynchronous transaction.

\endinput




https://www.linkedin.com/pulse/pci-express-primer-3-transaction-layer-simon-southwell/